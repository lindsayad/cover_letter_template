My two primary passions are software development and simulation of multi-physics problems. To witness my passion for both code development and physics, I have created an application named \href{https://github.com/lindsayad/zapdos}{Zapdos} that sits on top of the MOOSE framework which is capable of simulating problems in the field of low-temperature plasmas with possible extension to fusion plasmas. Our first manuscript based on Zapdos simulation results was recently accepted for publication in the Journal of Physics D. I have also made significant code contributions to MOOSE itself, including one 1,400 line pull request that implemented an entirely new system (interface\_kernels) for interfacing physics across subdomains as well as subdomain restriction of discontinuos Galerkin kernels. This software development has given me experience in version control (git), developing unit tests, and submitting clean, clear pull requests. While developing software tools, I have also been able to make novel contributions to plasma research as indicated by three primary authorships and a secondary authorship. In addition to computing and plasma experience, I have a strong nuclear background as indicated by my perfect marks in my degree program at North Carolina State University, the 6$^{th}$ rated school for nuclear engineering graduate programs.

In addition to learning the science behind industrial plasmas and nuclear reactors, my graduate studies have given me significant experience in numerical mathematics related to the solution of partial differential equations. This includes extensive work with finite element methods, both continuous and discontinuous Galerkin, low and high-order shape functions as well as solving the resulting non-linear equations using Newton's method. Use of Newton's method has entailed learning about line searches, direct LU factorization, iterative Krylov methods like GMRES, preconditioning, and Jacobian-free methods. I also have extensive experience with open source meshing and visualization software like gmsh and Paraview. I use sripting tools like bash and python to automate pre- and post-processing of data. Python library experience includes Sympy for symbolic calculations, Numpy for data manipulation, and MatPlotLib for generation of high-quality plots and figures. Python development work is handled with conda virtual environments.

I believe my extensive development work on MOOSE and my own MOOSE application, my knowledge of the numerical methods underlying the software, and python programming experience make me uniqely qualified for the advertised National Center for Supercomputing Applications (NCSA) post-doctoral position. Additionally, the group's commitment to open and transparent research (insomuch as it is possible) is incredibly appealing. As indicated by my open development of Zapdos, my committment to an open multiphysics platform (MOOSE), and use of python, I am a passionate advocate for openness in software and scientific research. Though I have received post-doctoral offers from several other institutions addressing interesting research subjects, I can honestly say that the NCSA position is more intriguing because of the collaborative culture and the chance to be surrounded by others who believe in open scientific computation. I believe I stand to both learn and contribute an immense amount. It is my goal to be a leader in the open scientific computing community; I believe the NCSA position will put me in the best position to help realize that dream.
